\documentclass{article}
\usepackage[utf8]{inputenc}
\usepackage[english]{babel}
\usepackage[letterpaper,top=1in,bottom=1in,left=1in,right=1in,marginparwidth=0in]{geometry}
\usepackage[colorlinks=true, allcolors=blue]{hyperref}
\usepackage{amsmath, graphicx, amssymb, amsthm, dsfont, tabu, subfigure, tikz, longdivision, array, booktabs}

\title{Project 2: Simulation of a Network Router Queue}
\author{Arian Djahed}
\date{December 10, 2024}

\begin{document}

\maketitle

\section{Simulation Design}

The six python files that make up my whole simulation architecture consist of five separate classes for the various objects referenced throughout the simulation and another python script with the actual simulation logic and with the logger that outputs the data we are trying to procure.

The first of these classes is \verb|Engine.py|, which—as the name suggests—simulates the routing engine that the packets go to once they leave the queue. It comes complete with parameters to indicate the current packet, the service time, and whether or not the engine is busy (along with getters and setters for each of these). There is also the \verb|Packet.py| and \verb|Event.py| classes; the former has parameters for packet ID, packet size, arrival time, and dequeue time; while the latter has parameters for event time and event type (and each have getters and setters for their respective parameters).

Then, there is the \verb|PacketQueue.py| class, which specifically calls upon the aforementioned \verb|Packet| class to make a queue specifically for packets and no other object. It comes complete with dedicated \verb|enqueue| and \verb|dequeue| functions that properly increment/decrement the number of items in the queue. I also designed it such that one could choose between either having an unlimited packet queue size or a limited packet queue size; one could achieve the former by setting \verb|maxItems| to zero when instantiating the object, and one could achieve the latter by setting the same variable to any positive integer. The \verb|EventQueue.py| class is similar, except it calls upon the \verb|Event| class to make a queue specifically for events; as such, there is no limit to the queue by default as this was never a requirement for the event queue. In addition, here, the \verb|enqueue| function behaves like a priority queue while the \verb|dequeue| function behaves like a regular queue, as per the instructions.

Lastly, there is the \verb|QueueSim.py| file, which is where the actual simulation happens, more specifically within the aptly-named \verb|run_simulation| function. Upon running this file, which also imports the classes from all other files, the user is prompted to choose which test case to run between the three under which we were instructed to simulate our code. Then, thanks to the imported \verb|logging| library, the results are then sent to \verb|QueueSim.log| instead of printing them straight out onto the terminal. In all 3 of the test cases, the service rate is kept at a constant 100 bits per second, while the arrival rate is 0.001 packets per microsecond in the first case, 0.01 packets per microsecond in the second, and 0.1 packets per microsecond in the third.

\section{Algorithms Used}

Here, I will give ``General Algorithm I" (as it is called in the instructions) as a pseudocode version of my python implementation in \verb|QueueSim.py|:

\subsection{General Algorithm I}

\begin{verbatim}
    FUNCTION run_simulation(num_packets, arrival_rate, service_rate)
        INITIALIZE event_queue, packet_queue, and engine objects
        
        SET total_wait_time to 0
        SET total_busy_time to 0
        SET max_queue_size to 0
        SET longest_wait_time to 0
        SET total_bytes to 0
        
        SET current_time to 0
        SET arrival time based on arrival_rate
        ENQUEUE new event based on arrival time onto event_queue
        SET processed_packets to 0

        WHILE processed_packets is less than num_packets
            DEQUEUE event from event_queue
            
            IF event_queue is empty
                SET arrival time based on arrival_rate
                ENQUEUE new event based on arrival time onto event_queue
            END IF

            IF current event type is "arrival"
                PRINT that a new packet has arrived
                INITIALIZE new packet object with random packet size
                ENQUEUE new packet to packet_queue
                SET new arrival time
                ENQUEUE new event based on arrival time onto event_queue

                IF engine is NOT busy
                    DEQUEUE next packet to be serviced from packet_queue
                    SET service_time to packet size divided by service_rate
                    ENQUEUE new event based on service_time onto event_queue
                    SET engine's busy status to TRUE
                END IF

            ELSE IF current event type is "departure"/"service completed"
                SET processed_packet to the current packet in the engine
                SET wait_time to processed_packet's dequeue time minus its arrival time
                SET longest_wait_time to the maximum between itself and wait_time
                ADD wait_time to total_wait_time
                ADD engine's service time to total_busy_time
                ADD processed_packet's size to total_bytes
                ADD one to processed_packets

                IF packet_queue is not currently empty:
                    DEQUEUE next packet to be serviced from packet_queue
                    SET service_time to packet size divided by service_rate
                    ENQUEUE new event based on service_time onto event_queue
                ELSE SET engine's busy status to FALSE
                END IF

                PRINT that the packet is finished
            END IF
        END WHILE

        SET total_time to total_wait_time plus total_busy_time
        SET avg_wait_tme to total_wait_time over num_packets
        SET utilization_rate to total_busy_time over current_time

        RETURN total_time, item history of packet_queue, total_bytes, max_queue_size,
               longest_wait_time, avg_wait_time, total_busy_time, utilization
    END FUNCTION
\end{verbatim}

\section{Sample Runs}

Since I designed my program so that all the results would appear in a separate \verb|.log| file, my sample runs will consist of screenshots of this file.

\subsection{Test Case 1: Arrival Rate \textless\textless{ Service Rate}}

\begin{center}
    \includegraphics[width=\textwidth]{Screen Shot 2024-12-08 at 2.59.22 PM.png}
\end{center}

\subsection{Test Case 2: Arrival Rate close to Service Rate}

\begin{center}
    \includegraphics[width=0.8125\textwidth]{Screen Shot 2024-12-08 at 3.07.52 PM.png}
\end{center}

\subsection{Test Case 3: Arrival Rate {\textgreater\textgreater} Service Rate}

\begin{center}
    \includegraphics[width=0.8125\textwidth]{Screen Shot 2024-12-08 at 3.16.01 PM.png}
\end{center}

\section{Results}

\subsection{Data Table}
Note: here $r_a$ refers to the arrival rate and $r_s$ refers to the service rate.

\begin{center}
\begin{tabular}{l|l|l|l}
                                      & $r_a << r_s$ & $r_a \approx r_s$ & $r_a >> r_s$ \\ \hline
Total Time ($\mu$s)                       & 596,675.44                                     & 5,463,623.97                                     & 27,628,325.93                                        \\
Total number of packets served        & 1000                                           & 1000                                             & 1000                                                 \\
Total Bytes Processed                 & 755,199                                        & 749,189                                          & 742,530                                              \\
Most packets held by queue            & 12                                             & 197                                              & 10,957                                               \\
Longest wait time for any packet ($\mu$s) & 6,486.02                                       & 10,795.64                                        & 54,700.01                                            \\
Average Wait Time ($\mu$s)                & 536.26                                         & 5,403.69                                         & 27,568.92                                            \\
Total time the engine was busy ($\mu$s)   & 60,415.92                                      & 59,935.12                                        & 59,402.40                                            \\
Router Utilization (\%)               & 11.90                                          & 99.24                                            & 99.98                                               
\end{tabular}
\end{center}

\subsection{Comments}

The first thing that seems to be of note is the fact that as the arrival rate increases by orders of magnitude, the total time increases by the same order of magnitude. It can also be seen that as the arrival rate increases but the service rate stays the same, the engine seems to get ``overwhelmed" as a result of the packets coming in quicker but the engine not being able to service them any quicker; this is reflected in the total packets held by the queue increasing astronomically as the arrival rate increases. In addition, the fact that the longest and average wait time increase as the service rate increases also reflects this.

What is most interesting, though, is the fact that the total bytes processed and the total time the engine was busy remain virtually unchanged even as the arrival rate increases (besides some negligible perturbations). This most likely indicates that these are independent of the arrival rate. Furthermore, in the case of the total time that the engine was busy, it reveals that increasing the arrival rate whilst keeping the service rate constant—and thus ``overwhelming" the engine—only affects the wait time between packet processing and not the actual time that it takes to process each packet. This, however, did affect the router utilization rate.

\end{document}
